% In this file you should put the actual content of the blueprint.
% It will be used both by the web and the print version.
% It should *not* include the \begin{document}
%
% If you want to split the blueprint content into several files then
% the current file can be a simple sequence of \input. Otherwise It
% can start with a \section or \chapter for instance.
\begin{lemma}\label{sum_pow}
	\lean{sum_pow}
	\leanok
	
	Given $n \in \mathbb N$ and $x \in \mathbb R$ with $x \ne 1$, we have
    $\sum_{i=0}^{n-1} x^i = \frac{x^n - 1}{x - 1}$.
\end{lemma}

\begin{lemma}\label{sum_pow_interest}
	\lean{sum_pow_interest}
	\leanok
	\uses{sum_pow}
	Given $n\in\mathbb N$ and $i\in\mathbb R$ with $i \ne 0$ and $1 + i \ne 0$,
  we have $\sum_{k=0}^n (1+i)^{-k}-1= \frac{1 - (1 + i)^{-n}}i$.
\end{lemma}

\begin{lemma}\label{id_mul_geom_sum}
\lean{id_mul_geom_sum}
\leanok
 Given (x : ℝ) (hx : x ≠ 1) (n : ℕ), we have
  $∑ k ∈ Finset.range (n+1), k * x^k =
  (x * (n * x^(n + 1) - ((n + 1) * x^n) + 1))/(x - 1)^2$.
\end{lemma}

\begin{lemma}\label{sum_Icc_succ_eq_sum_range}
	\lean{sum_Icc_succ_eq_sum_range}
	\leanok
	Given (f : ℕ → ℝ) (n : ℕ), we have
  $f 0 + ∑ k ∈ Finset.Icc 1 n, f k
    = ∑ k ∈ Finset.range (n+1), f k$.
\end{lemma}

\begin{lemma}\label{id_mul_geom_sum₁}
	\lean{id_mul_geom_sum₁}
	\leanok
	\uses{sum_Icc_succ_eq_sum_range,id_mul_geom_sum}
Given	(x : ℝ) (hx : x ≠ 1) (n : ℕ), we have
	$∑ k ∈ Finset.Icc 1 n, k * x^k =
	  (x * (n * x ^ (n + 1) - ((n + 1) * x ^ n) + 1)) / (x - 1) ^ 2$.
\end{lemma}

\begin{definition}\label{geom_sum}
	\lean{geom_sum}
	\leanok
	$\mathrm{geom_sum}$ is the function : ℕ → ℝ → ℝ := fun n v =>
	  $∑ k ∈ Icc 1 n, v ^ k$.
\end{definition}

\begin{definition}\label{annuity.a}
	\lean{annuity.a}
	\leanok
	\uses{annuity.geom_sum}
	The present value of annuity function $a : \mathbb N \to \mathbb R \to \mathbb R$
	is defined by $a n i$ equals \verb!geom_sum! $n (1 + i)^{-1}$.
\end{definition}

Main result: if $r > i$ and $d < 1 + 1 / i$ then we can recover $n$ (maturity) from $D$ (duration) and $r$ and $i$.
But actually $d < 1 + 1 / i$ follows from $r > i$.
If $r < i$ then we can only if $d < 1 + 1 / i$, which indeed does not hold in the graph in Chan and Tse.
From \texttt{AriExtremum.lean}, when $r<i$ and $d>1+1/i$ (``deep discount'') there is a unique local extremum
and we cannot infer $n$ from $D$ except at the extremum.

\begin{tabular}{ l l l}
			  & $r<i$											& $r > i$	\\
$d < 1 + 1/i$ &	can recover (\emph{unique\_solution})			& can recover (\emph{final\_result})\\
$d > 1 + 1/i$ &	cannot recover (\emph({AriExtremum.lean}, Chan and Tse figure))		& impossible by \emph{Aristotle\_duration.lean}\\
\end{tabular}

Combining these results the requirement is that $d < 1 + 1 / i$ overall.
Aristotle provided a single proof of that.

\begin{theorem}
	\label{eq_CPT_N_from_D_I}
	\lean{eq_CPT_N_from_D_I}
	\leanok
	
Given $n\in\mathbb N$ with $n>1$ and $i,d,r$ positive real numbers,	 with  $d < 1 + 1 / i$,
 if the duration equation holds for $n,i,r,d$ then $n$ is given by \verb!CPT_N_from_D i d r hd hi (by linarith) hdi!.
\end{theorem}

